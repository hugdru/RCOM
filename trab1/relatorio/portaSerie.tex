\documentclass[a4paper]{article}

\usepackage[portuguese]{babel}
\usepackage[utf8]{inputenc}
\usepackage{indentfirst}
\usepackage{graphicx}
\usepackage{verbatim}
\usepackage[T1]{fontenc}

\begin{document}

\setlength{\textwidth}{16cm}
\setlength{\textheight}{22cm}

\title{\Huge\textbf{Protocolo de Ligação de Dados}\linebreak\linebreak\linebreak
\Large\textbf{Relatório \\ Trabalho1}\linebreak\linebreak
\includegraphics[height=6cm, width=7cm]{feup.pdf}\linebreak \linebreak
\Large{Mestrado Integrado em Engenharia Informática e Computação} \linebreak \linebreak
\Large{Redes de Computadores}
}

\author{Hugo Ari Rodrigues Drumond --- 201102900 --- hugo.drumond@fe.up.pt \\ José Pedro Pereira Amorim --- 201206111 --- ei12190@fe.up.pt \\ João Ricardo Pintas Soares --- 201200740 --- ei12039@fe.up.pt\linebreak\linebreak\linebreak \\
 \\ Faculdade de Engenharia da Universidade do Porto \\ Rua Roberto Frias, 4200--65 Porto, Portugal \linebreak\linebreak\linebreak
\linebreak\linebreak\vspace{1cm}}
\maketitle
\thispagestyle{empty}

\newpage

\section{Introdução}
%indicação dos objectivos do trabalho e do relatório; descrição da lógica do relatório com indicações sobre o tipo de informação que poderá ser encontrada em cada uma secções seguintes
Este trabalho laboratorial, desenvolvido no âmbito da Unidade Curricular de Redes de Computadores (RCOM), teve como objetivo implementar um protocolo de ligação de dados, do tipo acknowledged connection-oriented, e testá-lo em diversas situações de stress de modo a verificar a sua robustez.
Ao longo deste relatório, serão descritos os aspetos fundamentais do referido trabalho, permitindo obter um conhecimento detalhado deste. Será apresentada a arquitetura, estrutura do código, casos de uso principais, protocolo de ligação lógica e de aplicação. No mesmo sentido, serão apresentadas a validação dos resultados e os elementos de valorização.

\section{Arquitectura}
%blocos funcionais e interfaces

\section{Estrutura do código}
%APIs, principais estruturas de dados, principais funções e sua relação com a arquitetura
\centerline{\includegraphics[scale=0.70]{organizacaoFicheirosECodigo.png}}

\section{Casos de uso}
%identificação; sequências de chamada de funções
O utilizador só pode interagir com a nossa aplicação através da linha de comandos. Criámos um caso de usos bastante completo, que permite ao utilizador mudar todas as opções do linkLayer e da applicacationLayer, que têm impacto na transferência de dados. Para além disso, o parser aceita um número ilimitado de portas séries quer sejam destinadas a receber ou enviar, e as respetivas configurações. Cada uma separada com o símbolo +. Para ver quais as opções que a nossa aplicação suporta basta corrê-la com a opção -h, serius -h. Também incluímos alguns exemplos de uso. Lá podemos encontrar o nome Bundle que basicamente significa um conjunto de opções (Options:) para uma dada porta série que opera de uma dada maneira (Mode:), receptor ou emissor. Por exemplo:\\\newline serius -d'/dev/tty100' -b115200 -t4 -r10 -f150 -s90 -S'pinguim.gif' + -d'/dev/tty200' -S'pinguim.gif'\\\newline
Neste exemplo, os seguintes settings são aplicados na porta série tty100: baudrate, timeout, retries, tamanho máximo do payload (unstuffed) e o tamanho máximo da parte dos dados do pacote de informação. E na tty200 os defaults são usados para todas essas opções e para as restantes. Em ambos os casos as portas série atuam como emissoras e enviam o ficheiro pinguim.gif para possivelmente computadores diferentes. Um dos computadores receptores teria um processo serius iniciado com os seguintes argumentos:\\\newline serius -d '/dev/tty300' -b 115200 -D\\ Cria um ficheiro com o nome que vem no pacote de controlo start, e coloca lá a informação que recebe. \\\newline E o outro:\\\newline serius -d '/dev/tty400' -R 'received.gif' \\\newline
De início, tínhamos em mente correr cada Bundle numa thread. Possibilitando transferir/receber de várias fontes simultâneamente via porta série, no entanto, não houve tempo para o fazer. O mesmo sucedeu com as pipes e a redireção da shell, cuja ideia era respetivamente: enviar informação acabada de ser processada; e, método alternativo de guardar ficheiros ou de os receber. \\\newline
\textbf{Diagrama de casos de uso:} \\\newline

\includegraphics[scale=0.5]{useCases.png}

\section{Protocolo de ligação lógica}
%identificação dos principais aspectos funcionais; descrição da estratégia de implementação destes aspectos com apresentação de extratos de código

\section{Protocolo de aplicação}
%identificação dos principais aspectos funcionais; descrição da estratégia de implementação destes aspectos com apresentação de extractos de código

\section{Validação}
%descrição dos testes efectuados com apresentação quantificada dos resultados, se possível

\section{Elementos de valorização}
%identificação dos elementos de valorização implementados; descrição da estratégia de implementação com apresentação de pequenos extratos de código

\section{Conclusões}
%síntese da informação apresentada nas secções anteriores; reflexão sobre os objectivos de aprendizagem alcançados

\end{document}
