\documentclass[a4paper]{article}

\usepackage[portuguese]{babel}
\usepackage[utf8]{inputenc}
\usepackage{indentfirst}
\usepackage{graphicx}
\usepackage{verbatim}
\usepackage[T1]{fontenc}

\begin{document}

\setlength{\textwidth}{16cm}
\setlength{\textheight}{22cm}

\title{\Huge\textbf{Protocolo de Comunicação}\linebreak\linebreak\linebreak
\Large\textbf{Relatório \\ Trabalho1}\linebreak\linebreak
\includegraphics[height=6cm, width=7cm]{feup.pdf}\linebreak \linebreak
\Large{Mestrado Integrado em Engenharia Informática e Computação} \linebreak \linebreak
\Large{Redes de Computadores}
}

\author{Hugo Ari Rodrigues Drumond \\ 201102900 --- hugo.drumond@fe.up.pt \\ José Pedro Pereira Amorim \\ 201206111 --- ei12190@fe.up.pt \\ João Ricardo Pintas Soares \\ 201200740 --- ei12039@fe.up.pt\linebreak\linebreak\linebreak \\
 \\ Faculdade de Engenharia da Universidade do Porto \\ Rua Roberto Frias, 4200--65 Porto, Portugal \linebreak\linebreak\linebreak
\linebreak\linebreak\vspace{1cm}}
\maketitle
\thispagestyle{empty}

\newpage

\section{Introdução}
%indicação dos objectivos do trabalho e do relatório; descrição da lógica do relatório com indicações sobre o tipo de informação que poderá ser encontrada em cada uma secções seguintes
Este trabalho laboratorial teve como objetivo implementar um protocolo de ligação de dados e testá-lo com aplicação de transferência de ficheiros, elaborada no âmbito da unidade curricular de Redes de Computadores.
Ao longo deste relatório, serão descritos os aspetos fundamentais do referido trabalho, permitindo obter um conhecimento detalhado deste. Será apresentada a arquitetura, estrtutura do código, casos de uso principais, protocolo de ligação lógica e de aplicação. No mesmo sentido, serão apresentadas a validação dos resultados e os elementos de valorização.

\section{Arquitectura}
%blocos funcionais e interfaces

\section{Estrutura do código}
%APIs, principais estruturas de dados, principais funções e sua relação com a arquitetura

\section{Casos de uso}
%identificação; sequências de chamada de funções

\section{Protocolo de ligação lógica}
%identificação dos principais aspectos funcionais; descrição da estratégia de implementação destes aspectos com apresentação de extratos de código

\section{Protocolo de aplicação}
%identificação dos principais aspectos funcionais; descrição da estratégia de implementação destes aspectos com apresentação de extractos de código

\section{Validação}
%descrição dos testes efectuados com apresentação quantificada dos resultados, se possível

\section{Elementos de valorização}
%identificação dos elementos de valorização implementados; descrição da estratégia de implementação com apresentação de pequenos extratos de código

\section{Conclusões}
%síntese da informação apresentada nas secções anteriores; reflexão sobre os objectivos de aprendizagem alcançados

\end{document}
